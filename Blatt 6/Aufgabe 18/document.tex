\documentclass{article}
\usepackage[utf8]{inputenc}
\usepackage[document]{ragged2e}
\begin{document}
\textbf{Aufgabe 18} \newline
\textit{a)}
Ausdruck 2 ist logisch nicht äquivalent. Dies ist einfach daran zu sehen, dass bei den anderen drei Ausdrücken stets w in jedem Term als multiplikativer Faktor vorhaden ist - also w stets wahr sein muss, damit der gesamte Ausdruck wahr ist. Dies ist bei 2 klar nicht der Fall. \linebreak

\textit{b)} Die logische Äquivalenz von 1 und 3 ist leicht durch Anwendung des Distributivgesetzes zu sehen. Zum Vergleich von 3 und 4 betrachten wir nun nur die Fälle, in denen w bereits wahr ist (da sonst eindeutig beide Terme falsch sind, siehe a)). \newline
3 ist wahr, genau wenn x falsch oder y falsch oder z falsch ist. 4 ist durch den letzten Oder-Ausdruck ebenfalls immer wahr, wenn z wahr ist.
Damit bleibt noch die Betrachtung über x und y. \newline
Der dritte Oder-Ausdruck in 4 zeigt, dass 4 genau dann wahr ist, wenn y falsch ist (denn ist z wahr, und damit der dritte Ausdruck falsch, ist der vierte Ausdruck wahr.) \newline
Zusätzlich ist 4 genau dann wahr, wenn x falsch ist - ist z wahr, was den ersten Oder-Ausdruck falsch setzt, so ist der vierte Ausdruck, und damit ganz 4 wahr. Ist y falsch, was ebenfalls den ersten Ausdruck falsch setzt, so ist der zweite Ausdruck, und damit der ganze Term 4), wahr. Somit sind auch 3 und 4 logisch äquivalent.


	



\end{document}