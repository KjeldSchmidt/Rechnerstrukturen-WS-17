\documentclass{article}
\begin{document}
\begin{flushleft}
\textbf{A15}
\end{flushleft}

\begin{tabular}[]{l c }
	(0) &  $abcdef\overline{g}$ \\
	(1) & $+\overline{a}b\overline{cdefg}$\\
	(2) & $+ab\overline{c}de\overline{f}g$\\
	(3) & $+abcd\overline{ef}g$\\
	(4) & $+\overline{a}bc\overline{de}fg$\\
	(5) & $+a\overline{b}cd\overline{e}fg$\\
	(6) & $+a\overline{b}cdefg$\\
	(7) & $+abc\overline{defg}$\\
	(8) & $+abcdefg$\\
	(9) & $+\overline{abcd}e\overline{fg}$\\
\end{tabular}
\linebreak
\begin{flushleft}
	Diese DNF beschreibt genau die gueltigen Zustaende der Anzeige, die nur Ziffern erlaubt. \linebreak
\end{flushleft}

\newcommand{\NANDNEG}[1]{#1 - #1}
\newcommand{\NANDAND}[2]{ (#1 - #2) - (#1 - #2) }
\newcommand{\NANDOR}[2]{(#1 - #1) - (#2 - #2)}
\newcommand{\NEG}[1]{\overline{#1}}
\newcommand{\AND}[2]{#1\cdot #2}
\newcommand{\OR}[2]{#1 + #2}
\pagebreak
\begin{flushleft}
\textbf{A16a}

Sei NAND(X, Y) = X-Y \linebreak
Hier nicht gezeigt sind Wahrheitstafeln die beweisen: \linebreak
$X-X = \bar{X}$ = NANDNEG \linebreak
$(X-Y)-(X-Y) = X*Y$ = NANDAND \linebreak
$\bar{X} - \bar{Y}$ = X+Y = NANDOR \linebreak


So zeigt sich \linebreak
\end{flushleft}
$\OR
	{\AND{x_1}{\NEG{x_2}}}
	{\NEG{
		\OR
			{\AND{x_2}{x_3}}
			{\OR
				{\NEG{x_1}}
				{x_3}
			}
		}
	}
$
 = 
$\NANDOR
{\NANDAND{x_1}{\NANDNEG{x_2}}}
{\NANDNEG{
		\NANDOR
		{\NANDAND{x_2}{x_3}}
		{\NANDOR
			{\NANDNEG{x_1}}
			{x_3}
		}
	}
}
$

\begin{flushleft}
\textbf{A16b} \linebreak
Zum zeigen von Funktionaler Vollstaendigkeit von $\Rightarrow$ und der konstanten Nullfunktionen leiten wir aus diesen beiden die als funktional vollstaendig bekannte Funktionsmenge UND, ODER, NEGATION (*, +, $\bar{ }$) ab. \linebreak
Unter verwendung von Wahrheitstafeln laesst sich leicht zeigen:

$\bar{ }(x) = (x \Rightarrow 0 )$\linebreak
$+(x, y) = \bar{x} \Rightarrow y$\linebreak
$*(x, y) = \overline{(x \Rightarrow \bar{y})}$\linebreak

\end{flushleft}
\pagebreak
\begin{flushleft}
\textbf{A17} \linebreak

Da die Eingabe als Binaerzahl zu interpretieren ist, können wir feststellen, dass die hoechste darstellbare Zahl 31 ist. Somit gibt es keinen Fall, in dem die Eingabe durch 7 \textit{und} 5 teilbar ist (kgV(7, 5) = 35). Somit ist:

$f^{-1} = \{00101, 01010, 01111, 10100, 11001, 11110, 01110, 10101, 11100 \} = $
$f^{-1} = \{ 5, 10, 15, 20, 25, 30, 7, 14, 21 \}$

Der exakte Funktionsterm von $f$ ist die DNF die sich bildet, wenn man die individuellen Binaerdarstellungen mit ODER verknuepft, die Binaerziffern mit UND und jede 0 negiert. Dies wird hier nicht explizit umgesetzt.

\end{flushleft}


\end{document}
